\input{preamble.tex}

\title{A template \LaTeX{} report}
\author{Author Authorsen}

\begin{document}

\maketitle

\abstract{This document demonstrates usage of \LaTeX{}.}

\thispagestyle{empty}

\clearpage
\pagenumbering{roman}
\setcounter{page}{1}
\tableofcontents

\clearpage
\pagenumbering{arabic}

% BEGINNING OF CONTENT

\section{Introduction}

This is a \LaTeX{} report template. It demonstrates cross-references, citations, images, and code examples with minted.

\section{Examples}

\subsection{Code}

The following is a demonstration of how cross-references can be used to refer to appendix code. Citations are also used.
\\
\\
Beaker can work with different hypervisors by using plugins. If a plugin for a particular hypervisor does not exist, an alternative is to use Vagrant to manage the SUT's, and instead install and run Beaker as part of Vagrant's provisioning process.~\cite{puppetmodulefunctionaltestingbeakerinsidevagrant}\cite{openstackpuppetmodulefunctionaltestingproposal} An example of this is included in appendix~\ref{beakerinsidevagrantexample}. This example would be used by running \mintinline[breaklines]{bash}{vagrant up --verbose && vagrant destroy --force --verbose}.

\subsection{Images}

Figure \ref{sunnmore} shows a picture of Sunnmørsalpane. You're welcome.

\begin{figure}[H]   % H forces the picture to show up under the text
\caption{Sunnmørsalpane}
\centering
\includegraphics[width=\textwidth]{sunnmore}
\label{sunnmore}
\end{figure}

\section{Conclusion}

This has been a demonstration of \LaTeX{} in use.

% BEGINNING OF REFERENCES

\clearpage % make references start on own page

% This makes LaTeX list all references in the bib file, instead of just 
% the ones that are cited with a \cite command
\nocite{*}

% BEGINNING OF APPENDIX

\bibliographystyle{acmdoi}
\bibliography{report}

\clearpage % make appendix start on own page
\appendix

\section{Beaker inside Vagrant example} \label{beakerinsidevagrantexample}

\begin{minted}[fontsize=\tiny,linenos,breaklines]{ruby}
# -*- mode: ruby -*-
# vi: set ft=ruby :

require 'vagrant-openstack-provider'

#
# This is quite the minimal configuration necessary
# to start an OpenStack instance using Vagrant on
# an OpenStack with Keystone v3 API
#
# NOTE: this example is heavily
# inspired by http://my1.fr/blog/puppet-module-functional-testing-with-vagrant-openstack-and-beaker/
#
Vagrant.configure('2') do |config|

  config.ssh.username = 'ubuntu'

  config.vm.provider :openstack do |os, ov|
    os.server_name                      = 'vagrant_machine_in_openstack'
    os.security_groups                  = [ 'default', 'linux' ]
    os.identity_api_version             = '3'
    os.openstack_auth_url               = 'https://api.skyhigh.iik.ntnu.no:5000/v3'
    os.project_name                     = '<PROJECTNAME>'
    os.user_domain_name                 = 'NTNU'
    os.project_domain_name              = 'NTNU'
    os.username                         = '<USERNAME>'
    os.password                         = '<PASSWORD>'
    os.region                           = 'SkyHiGh'
    os.floating_ip_pool                 = 'ntnu-internal'
    os.floating_ip_pool_always_allocate = true
    os.flavor                           = 'm1.small'
    os.image                            = 'Ubuntu Server 16.04 LTS (Xenial Xerus) amd64'
    os.networks                         = [ '<INTERNALNETID>' ]

    ov.nfs.functional = false
  end

  # you could provision this machine using the same provisioning scripts used by 
  # Heat, to create an exact duplicate
  config.vm.provision "shell", path: "bootscriptFromHeat.sh"
  

  # shell to install beaker, setup ssh, and run beaker tests.
  # written inline for sake of example
  config.vm.provision "shell", inline: <<-SHELL
    #!/bin/bash

    # install deps
    sudo apt-get update
    sudo apt-get install -y libxml2-dev libxslt-dev zlib1g-dev git ruby ruby-dev build-essential

    # prepare ssh
    echo "" | sudo tee -a /etc/ssh/sshd_config
    echo "Match address 127.0.0.1" | sudo tee -a /etc/ssh/sshd_config
    echo "    PermitRootLogin without-password" | sudo tee -a /etc/ssh/sshd_config
    echo "" | sudo tee -a /etc/ssh/sshd_config
    echo "Match address ::1" | sudo tee -a /etc/ssh/sshd_config
    echo "    PermitRootLogin without-password" | sudo tee -a /etc/ssh/sshd_config
    mkdir -p .ssh
    ssh-keygen -f ~/.ssh/id_rsa -b 2048 -C "beaker key" -P ""
    sudo mkdir -p /root/.ssh
    sudo rm /root/.ssh/authorized_keys
    cat ~/.ssh/id_rsa.pub | sudo tee -a /root/.ssh/authorized_keys
    sudo service ssh restart
   
    # prepare gems
    # this uses my gossinbacup module as an example, but it would be
    # possible to have the module as a parameter to this process
    git clone https://github.com/tholok97/gossinbackup
    cd gossinbackup
    sudo gem install bundler --no-rdoc --no-ri --verbose
    bundle install

    # run tests
    # this relies on SUT yaml definitions with hyporvisor set to "none",
    # like here: https://github.com/openstack/puppet-keystone/blob/master/spec/acceptance/nodesets/nodepool-xenial.yml
    export BEAKER_debug=yes
    bundle exec rspec spec/acceptance
SHELL

end
\end{minted}

\end{document}
